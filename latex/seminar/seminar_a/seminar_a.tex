\documentclass[xcolor]{beamer}

\usepackage[T1]{fontenc}

\usetheme{Malmoe}
\useoutertheme{infolines}
\useinnertheme{circles}

\makeatletter
\setbeamertemplate{mini frames}{}
\setbeamertemplate{headline}{%
	\begin{beamercolorbox}[ht=2.5ex,dp=1.125ex]{section in head/foot}
		\insertnavigation{\paperwidth}
	\end{beamercolorbox}%
}%
\makeatother
\setbeamertemplate{caption}[numbered]

% Colours defined here
\definecolor{DarkGreen}{HTML}{41964b}
\definecolor{LightGreen}{HTML}{46c755}
\definecolor{FG}{HTML}{ebebeb}
\definecolor{BG2}{HTML}{233628}
\definecolor{BG}{HTML}{202622}

\setbeamercolor{palette primary}{bg=BG, fg=FG}
\setbeamercolor{palette secondary}{bg=BG, fg=FG}
\setbeamercolor{palette tertiary}{bg=BG, fg=FG}
\setbeamercolor{palette quaternary}{bg=BG, fg=FG}

\setbeamercolor{section in head/foot}{bg=BG2, fg=DarkGreen}
\setbeamercolor{subsection in head/foot}{bg=BG2, fg=LightGreen}
\setbeamercolor{page number in head/foot}{bg=BG2, fg=LightGreen}
\setbeamercolor{author in head/foot}{bg=BG2, fg=DarkGreen}
\setbeamercolor{date in head/foot}{bg=BG2, fg=DarkGreen}
\setbeamercolor{title in head/foot}{bg=BG2, fg=LightGreen}

\setbeamercolor{normal text}{fg=FG, bg=BG}
\setbeamercolor{structure}{fg=LightGreen, bg=BG}

\setbeamerfont{subsection in toc}{size=\scriptsize}
\setbeamerfont{section in toc}{size=\footnotesize}

\usefonttheme{serif}

\usepackage{unicode-math}
\setmainfont{Gentium Basic} % Main font
\setmonofont{Fantasque Sans Mono} % Code font

\usepackage{amsmath}
\usepackage{amssymb}
\usepackage{microtype}

\usepackage{listings}
\lstset{%
	language=C,
	frame=single,
	backgroundcolor=\color{BG2},
	basicstyle={\tiny\ttfamily\color{FG}},
	keywordstyle=\color{LightGreen}
}

\usepackage{graphicx}
\graphicspath{{./files/}}

\usepackage{tikz}
\usetikzlibrary{shapes.geometric, positioning}
% tikz style
\tikzstyle{startstop} = [rectangle, rounded corners, minimum width=2cm, minimum height=1cm,text centered, draw=white]
\tikzstyle{io} = [trapezium, trapezium left angle=70, trapezium right angle=110, minimum width=2cm, minimum height=0.75cm, text centered, draw=white]
\tikzstyle{process} = [rectangle, minimum width=2cm, minimum height=1cm, text centered, draw=white]
\tikzstyle{decision} = [diamond, minimum width=2cm, minimum height=1cm, text centered, draw=white]

\usepackage{cancel}

\author[Kyu-Sang Kim]{Kyu-Sang Kim\\z5208931\\\vspace{0.2cm}Supervised by Andrew Taylor (UNSW)\\Assessed by John Shepherd (UNSW)}
\title[Thesis A Seminar]{A Testing Tool for Introductory Programming Courses}
\subtitle{Thesis A Seminar}
\date{Term 1, 2022}

\begin{document}

\begin{frame}
	\titlepage
\end{frame}

\AtBeginSection[]{
	\begin{frame}
		\frametitle{Contents}

		\centering
		\begin{columns}
			\begin{column}{0.75\textwidth}
				\tableofcontents[currentsection]
			\end{column}
			\begin{column}{0.15\textwidth}
				% Use this for anything you want to display alongside the contents
			\end{column}
		\end{columns}
	\end{frame}
}

\section{Introduction}
\subsection{Motivation - Student Enrolments in Introductory Courses}
\begin{frame}
	\frametitle{Student Enrolments in Introductory Courses}
	Student numbers enrolling in introductory programming courses at UNSW have significantly increased in recent years\\
		\pause
	\begin{table}[h!]
		\centering
		\begin{tabular}{|l|c|c|c|c|c|}
			\hline
			\textbf{Enrolments/Year} & \textbf{2017} & \textbf{2018} & \textbf{2019} & \textbf{2020} & \textbf{2021} \\ 
			\hline
			\textbf{COMP1511} & 1351 & 1655 & 1874 & 1905 & 2381\\ 
			\hline
			\textbf{COMP1521} & 715 & 1136 & 1352 & 1417 & 1633\\
			\hline
			\textbf{COMP2521} & 378 & 1019 & 1389 & 1445 & 1551\\
			\hline
		\end{tabular}
		\caption{Course enrolments from UNSW Class Timetable (April 2022)}
		\label{table:1}
		\vspace{-4mm}
	\end{table}
		\pause
	The trend in increasing student enrolments for these courses present two main challenges for course staff:\\
		\pause
	\begin{enumerate}
		\item How can courses feasibly start/continue utilising practical assessments without automation tools as the staff to student ratio decreases?
			\pause
		\item How much of the marking process can be automated via tools?
			\pause
		\item Will the automation tools remain viable as assessments change?
	\end{enumerate}
\end{frame}

\subsection{Motivation - Andrew Taylor autotest}
\begin{frame}
	\frametitle{Andrew Taylor autotest}
	\begin{enumerate}
		\setlength\itemsep{1em}
		\item Written by Andrew Taylor in 2015 (Python rewrite)
			\pause
		\item First introduced to UNSW in COMP2041 2015 S2
			\pause
		\item Sufficiently capable and in current use to perform automated marking for most UNSW introductory programming courses
			\pause
		\item Has some \textbf{design deficiencies} and accumulated non-trivial \textbf{technical debt} since introduction
			\pause
		\item Deeper exploration of architecture and implementation details in Background section of Seminar
	\end{enumerate}
\end{frame}

\subsection{Motivation - Technical Debt}
\begin{frame}
	\frametitle{Technical Debt}
	\begin{enumerate}
		\setlength\itemsep{1em}
		\item We approach technical debt as outlined in the framework defined within \textit{An Exploration of Technical Debt}\footnote{Tom Edith, Aybüke Aurum and Richard Vidgen, ‘An Exploration of Technical Debt’ (2013) 86(6) \textit{The Journal of systems and software 1498}}
			\pause
		\item Cunningham\footnote{Ward Cunningham, ‘The WyCash Portfolio Management System’ in \textit{Addendum to the Proceedings on Object-Oriented Programming Systems, Languages, and Applications (addendum)} (ACM, 1992) 29}, who introduced the concept of technical debt, described how \textit{“shipping first time code is like going into debt. A little debt speeds development so long as it is paid back promptly with a rewrite”}
			\pause
		\item How can we reconcile existing technical debt and ensure debt remains manageable in the future?
	\end{enumerate}
\end{frame}
\begin{frame}
	\frametitle{Approaches to Managing Technical Debt}
	\begin{enumerate}
		\setlength\itemsep{0.5em}
		\item Extensive Project \& Code Documentation
			\pause
		\item Project Management Tools - GitHub, Jira etc.
			\pause
		\item Modular Architectural Design
			\pause
		\item Modernisation of Solution - Feature extensibility etc.
			\pause
		\item Automated Regression Testing
			\pause
	\end{enumerate}
	~\\
	Introducing all of these technical debt management solutions to the existing Andrew Taylor autotest software package has been determined to be infeasible due to the extensive refactoring required.
	\\~\\
	Thus, we propose that a new software package be created from a clean slate to minimise potential sources for both current and future technical debt from the ground up.
\end{frame}

\subsection{Thesis Statement}
\begin{frame}
	\frametitle{Thesis Statement}
	\textbf{A user-friendly and maintainable general code testing tool is important to streamline the administration of introductory programming courses}
	\\~\\
		\pause
	We will:
		\pause
	\begin{enumerate}
		\item Develop an extensible and easy to use software package which parses and runs pre-written tests on submitted code
			\pause
		\item Implement development procedures that minimise both current and future technical debt
			\pause
		\item Remediate known flaws in the existing autotest package
			\pause
		\item Maintain backwards compatibility with legacy tests written for the existing autotest package
			\pause
		\item Perform proving and performance tests on the new software package
			\pause
		\item Deprecate and replace the existing autotest used for introductory programming courses at UNSW CSE
	\end{enumerate}
\end{frame}

\section{Literature Review}
\subsection{Background - Code Marking in Introductory Courses}
\begin{frame}
	\frametitle{Code Marking in Introductory Courses}
	\begin{enumerate}
		\setlength\itemsep{1em}
		\item We assume that \textit{code marking} refers to the determination of whether submitted code when run conforms to some behaviour that is outlined in a specification or any similar resource
			\pause
		\item What are some potential issues with manual code marking?\footnote{Jackson David, ‘Using Software Tools to Automate the Assessment of Student Programs’ (1991) 17(2) \textit{Computers and education} 133}
			\pause
		\begin{itemize}
			\item Tedious and repetitive work is prone to mistakes by human error
				\pause
			\item Potential inconsistencies when distributed between separate markers
				\pause
			\item Increasing marking load per course staff member due to increasing student numbers can exacerbate the aforementioned issues
				\pause
		\end{itemize}
		\item How can we eliminate or reduce these issues?
	\end{enumerate}
\end{frame}

\subsection{Background - Automated Testing \& Marking Approaches}
\begin{frame}
	\frametitle{Automated Testing \& Marking Approaches}
	\begin{enumerate}
		\setlength\itemsep{1em}
		\item The earliest example of automated testing on student code was published in 1960 by Jack Hollingsworth of the Rensselaer Polytechnic Institute\footnote{Jack Hollingsworth, ‘Automatic Graders for Programming Classes’ (1960) 3(10) \textit{Communications of the ACM 528}}
			\pause
		\item \textbf{Main goal:} Autonomously verify the correctness of student submitted code in relation to pre-defined expected behaviour
			\pause
		\item \textbf{Main benefit:} Automation of performance/correctness testing of student submitted code
		\begin{itemize} 
			\item Time spent on manual testing by staff can be utilised in other tasks
				\pause
			\item Students were observed to learn programming better with an automatic grader over dedicated lab groups 
				\pause
			\item Increasing enrolments for programming courses per teaching period becomes economically feasible 
		\end{itemize}
	\end{enumerate}
\end{frame}
\begin{frame}
	\frametitle{Automated Testing \& Marking Approaches}
	\begin{enumerate}
		\setlength\itemsep{1em}
		\item How can we methodically determine that \textit{``submitted code matches expected behaviour"}?
			\pause
		\begin{itemize} 
			\item External Program Side-effect Comparison
				\pause
			\item Internal Program Unit Testing
				\pause
			\item Combination of above\footnote{Soundous Zougari, Mariam Tanana and Abdelouahid Lyhyaoui, ‘Hybrid Assessment Method for Programming Assignments’ in \textit{2016 4th IEEE International Colloquium on Information Science and Technology (CiSt)} (IEEE, 2016) 564}
				\pause
		\end{itemize}
		\item What are the specifics of these methodologies and what are some existing automated testing frameworks?
	\end{enumerate}
\end{frame}
\begin{frame}
	\frametitle{External Program Side-effect Comparison}
	\begin{columns}
		\begin{column}{0.25\textwidth}
			\begin{tikzpicture}
				\node (Tester)[process]{Tester};
				\node (Program Output)[io, below = 2em of Tester, align=center]{Program\\Output};
				\node (Program)[process, below = 2em of Program Output]{Program};
				\node (Expected Output)[io, above = 2em of Tester, align=center]{Expected\\Output};
				\draw [-stealth] (Expected Output) -- (Tester);
				\draw [-stealth] (Program Output) -- (Tester);
				\draw [-stealth] (Program) -- (Program Output);
				\draw [-stealth, rounded corners] (Tester.east) -- (1.5,0) |- (Program.east);
			\end{tikzpicture}
		\end{column}
		\begin{column}{0.7\textwidth}
			\begin{enumerate}
				\setlength\itemsep{0.70em}
				\item Assume program side-effects to be \textbf{observable output} from a tested program which has been stored on external resources such as the console via \textit{stdout} or files that store logs/program state
					\pause
				\item This output can be \textbf{externally compared} with those pre-generated by a sample solution to determine whether expected behaviour has been achieved
					\pause
				\item Comparisons on side-effects after program execution allows for easier testing of complex programs with dependencies on multiple different components at the cost of storage space
			\end{enumerate}
		\end{column}
	\end{columns}
\end{frame}
\begin{frame}
	\frametitle{Internal Program Unit Testing}
	\begin{columns}
		\begin{column}{0.25\textwidth}
			\begin{tikzpicture}
				\node (Tester)[process]{Tester};
				\node (Program)[process, below = 2em of Tester]{Program};
				\node (Expected Output)[io, above = 2em of Tester, align=center]{Expected\\State};
				\draw [-stealth] (Expected Output) -- (Tester);
				\draw [-stealth] (Program) -- (Tester);
			\end{tikzpicture}
		\end{column}
		\begin{column}{0.7\textwidth}
			\begin{enumerate}
				\setlength\itemsep{0.75em}
				\item Assume unit testing refers to the testing of \textbf{individual discrete code components} within the program by comparing internal program state to a known expected state pre-generated by a sample solution
					\pause
				\item Executing and collecting the results of the internal unit tests for the program can be used to determine whether expected behaviour has been achieved
					\pause
				\item Testing internal code components allow for more thorough determination of program correctness at the cost of difficulty in testing more complex programs which may not be able to share the same unit testing framework
			\end{enumerate}
		\end{column}
	\end{columns}
\end{frame}

\subsection{Existing Work - Andrew Taylor autotest}
\begin{frame}
	\frametitle{Andrew Taylor autotest}
	\begin{enumerate}
		\setlength\itemsep{1em}
		\item \textbf{Author(s) \& Introduction:} Andrew Taylor (University of New South Wales) - 2015 
			\pause
		\item \textbf{Testing Methodology:} External Program Side-effect Comparison 
			\pause
		\item \textbf{Language:} Python (+ some Shell and Perl script interfaces) 
			\pause
		\item \textbf{Maintained/Support:} Not regularly maintained outside of rare UNSW CSE student engagement
			\pause 
		\item Current UNSW CSE automated general code testing tool utilised for lab and assessment marking in most introductory programming courses and at times, some higher level courses (COMP Level 2+) 
	\end{enumerate}
\end{frame}
\begin{frame}[fragile]
	\frametitle{Andrew Taylor autotest implementation details}
	\begin{columns}
		\begin{column}{0.55\textwidth}
			\centering
\begin{lstlisting}[caption={autotest Example Test Cases}]
files=is_prime.c

1 stdin="39" expected_stdout="39 is not prime\n"
2 stdin="42" expected_stdout="42 is not prime\n"
3 stdin="47" expected_stdout="47 is prime\n"
\end{lstlisting}
\begin{lstlisting}[language=bash, breaklines=true, caption={autotest Wrapper}]
#!/bin/sh

parameters="
default_compilers = {'c' : [['clang', '-Werror', '-std=gnu11', '-g', '-lm']]}
upload_url = https://example.com/autotest.cgi
"

exec <path_to_autotest>/autotest.py --exercise_directory <path_to_exercise_dir> --parameters "$parameters" "$@"
\end{lstlisting}
		\end{column}
		\begin{column}{0.35\textwidth}
			\centering
			\begin{tikzpicture}
				\node (Tester)[process]{autotest};
				\node (Program Output)[io, below = 2em of Tester, align=center]{Program\\Output};
				\node (Program)[process, below = 2em of Program Output]{Program};
				\node (Expected Output)[io, above = 2em of Tester, align=center]{Test Case};
				\draw [-stealth] (Expected Output) -- (Tester);
				\draw [-stealth] (Program Output) -- (Tester);
				\draw [-stealth] (Program) -- (Program Output);
				\draw [-stealth, rounded corners] (Tester.east) -- (1.5,0) |- (Program.east);
			\end{tikzpicture}
		\end{column}
	\end{columns}
\end{frame}
\begin{frame}
	\frametitle{Andrew Taylor autotest pros \& cons}
	\begin{enumerate}
		\item \textbf{Pros:}
		\begin{itemize}
			\item Proven to be \textit{mostly} reliable at UNSW CSE
			\item Can support any programming language assuming autotest can detect and compare side-effects
			\item Extensive parameters exposed to manage test execution environment
			\item Test results on failure provide meaningful information on the differences between actual and expected output
		\end{itemize}
			\pause
		\item \textbf{Cons:}
		\begin{itemize}
			\item High levels of technical debt due to outdated use of technology and architecture
			\item Significant lack of documentation
			\item Difficult for first-time users to create tests and manage test execution environment
		\end{itemize}
	\end{enumerate}
\end{frame}

\subsection{Existing Work - Harvard CS50 check50}
\begin{frame}
	\frametitle{Harvard CS50 check50}
	\begin{enumerate}
		\setlength\itemsep{1em}
		\item \textbf{Author(s) \& Introduction:} Chad Sharp (Harvard University) - 2012\footnote{Chad Sharp et al, ‘An Open-Source, API-Based Framework for Assessing the Correctness of Code in CS50’ in \textit{Proceedings of the 2020 ACM Conference on Innovation and Technology in Computer Science Education} (ACM, 2020) 487}
			\pause
		\item \textbf{Testing Methodology:} External Program Side-effect Comparison 
			\pause
		\item \textbf{Language:} Python
			\pause
		\item \textbf{Maintained/Support:} Regularly maintained with active development (in private repository)
			\pause 
		\item Current Harvard University CS50: Introduction to Computer Science automated general code testing tool for checking correctness of practical lab exercises
	\end{enumerate}
\end{frame}
\begin{frame}[fragile]
	\frametitle{Harvard CS50 check50 Implementation Details}
	\begin{columns}
		\begin{column}{0.55\textwidth}
			\centering		
\begin{lstlisting}[language=python, breaklines=true, caption={check50 tests}]
import check50
import check50.c

@check50.check()
def exists():
"""hello.c exists"""
check50.exists("hello.c")

@check50.check(exists)
def compiles():
"""hello.c compiles"""
check50.c.compile("hello.c", lcs50=True)

@check50.check(compiles)
def emma():
"""responds to name Emma"""
check50.run("./hello").stdin("Emma").stdout("Emma").exit()
\end{lstlisting}
		\end{column}
		\begin{column}{0.35\textwidth}
			\centering
			\begin{tikzpicture}
				\node (Tester)[process]{check50};
				\node (Test Runner)[process, below = 2em of Tester]{Test Runner};
				\node (Tests)[io, above = 2em of Tester, align=center]{Test Cases};
				\node (Program)[process, below = 2em of Test Runner]{Program};
				\draw [-stealth] (Tests) -- (Tester);
				\draw [stealth-stealth] (Tester) -- (Test Runner);
				\draw [stealth-stealth] (Test Runner) -- (Program);
			\end{tikzpicture}
		\end{column}
	\end{columns}
\end{frame}
\begin{frame}
	\frametitle{Harvard CS50 check50 Pros \& Cons}
	\begin{enumerate}
		\item \textbf{Pros:}
		\begin{itemize}
			\item Tests are very simple to create as a chain of functions
			\item Documentation is very extensive
			\item Testing results can be rendered to HTML via a module for easier viewing
			\item Supports running of tests on both local and remote machines (PaaS Support)
			\item Concurrent running of tests is supported 
		\end{itemize}
			\pause
		\item \textbf{Cons:}
		\begin{itemize}
			\item As a result of design choice for simplicity, testing of complex programs can be challenging (may require Harnessing)
			\item No official support or implementations for programming languages outside of C and Python (Flask Supported)
		\end{itemize}
	\end{enumerate}
\end{frame}

\subsection{Existing Work - Google gtest}
\begin{frame}
	\frametitle{Google gtest}
	\begin{enumerate}
		\setlength\itemsep{1em}
		\item \textbf{Author(s) \& Introduction:} Google - v1.0.0 released in 2008
			\pause
		\item \textbf{Testing Methodology:} Internal Program Unit Testing (Technically Supports External Program Side-effect Comparison)
			\pause
		\item \textbf{Language:} C++
			\pause
		\item \textbf{Maintained/Support:} Regularly maintained by Google and Open Source Community
			\pause 
		\item Originally created by Google for internal use but has become one of the most popular C++ unit testing frameworks within the xUnit family of testing frameworks
	\end{enumerate}
\end{frame}
\begin{frame}[fragile]
	\frametitle{Google gtest Implementation Details}
	\begin{columns}
		\begin{column}{0.45\textwidth}
			\centering
\begin{lstlisting}[language=c, breaklines=true, caption={gtest test cases}]
TEST(FactorialTest, Positive) {
    EXPECT_EQ(1, Factorial(1));
    EXPECT_EQ(2, Factorial(2));
    EXPECT_EQ(6, Factorial(3));
    EXPECT_EQ(40320, Factorial(8));
}

TEST(IsPrimeTest, Positive) {
    EXPECT_FALSE(IsPrime(4));
    EXPECT_TRUE(IsPrime(5));
    EXPECT_FALSE(IsPrime(6));
    EXPECT_TRUE(IsPrime(23));
}
\end{lstlisting}
		\end{column}
		\begin{column}{0.45\textwidth}
			\centering
			\begin{tikzpicture}
				\node (Tester)[process]{gtest Runner};
				\node (Program)[process, below = 2em of Tester]{Program};
				\node (Tooling)[process, below = 2em of Program]{Google Test Library};
				\draw [stealth-stealth] (Tester) -- (Program);
				\draw [-stealth] (Tooling) -- (Program);
			\end{tikzpicture}
		\end{column}
	\end{columns}
\end{frame}
\begin{frame}
	\frametitle{Google gtest Pros \& Cons}
	\begin{enumerate}
		\item \textbf{Pros:}
		\begin{itemize}
			\item Documentation is extensive with a large community
			\item Concurrent execution of tests is supported
			\item Performance benefits of C++
		\end{itemize}
			\pause
		\item \textbf{Cons:}
		\begin{itemize}
			\item Testing of complex systems can be difficult as per standard with Internal Program Unit Testing methodology
			\item Setting up test execution environment is not common for most users and can be difficult to configure based on testing needs
		\end{itemize}
	\end{enumerate}
\end{frame}

\subsection{Existing Work - Other \& Historic Software}
\begin{frame}
	\frametitle{JUnit}
	\begin{enumerate}
		\setlength\itemsep{1em}
		\item \textbf{Author(s) \& Introduction:} Kent Beck, Erich Gamma, David Saff, Kris Vasudevan - Initial Prototype in 1997\footnote{Martin Fowler, "Bliki: Xunit", \textit{martinfowler.com} (Webpage, 2022) <https://martinfowler.com/bliki/Xunit.html>}
			\pause
		\item \textbf{Testing Methodology:} Internal Program Unit Testing (Technically Supports External Program Side-effect Comparison)
			\pause
		\item \textbf{Language:} Java
			\pause
		\item \textbf{Maintained/Support:} Regularly maintained by JUnit team and Open Source Community
			\pause 
		\item A very popular open source unit testing framework for Java applications within the xUnit family of testing frameworks
	\end{enumerate}
\end{frame}
\begin{frame}
	\frametitle{BAGS - Basser Automatic Grading Scheme}
	\begin{enumerate}
		\setlength\itemsep{1em}
		\item \textbf{Author(s) \& Introduction:} J Hext and J Winings (University of Sydney) - Earliest Documented in 1968\footnote{J Hext and J Winings, ‘An Automatic Grading Scheme for Simple Programming Exercises’ (1969) 12(5) \textit{Communications of the ACM} 272}
			\pause
		\item \textbf{Testing Methodology:} Primitive External Program Side-effect Comparison
			\pause
		\item \textbf{Language:} Unknown (one of the KDF9 Operating System languages)
			\pause
		\item \textbf{Maintained/Support:} Unknown (Assumed to be abandoned due to age)
	\end{enumerate}
\end{frame}
\begin{frame}
	\frametitle{Kassandra}
	\begin{enumerate}
		\setlength\itemsep{1em}
		\item \textbf{Author(s) \& Introduction:} Urs von Matt (ETH Zürich) - Earliest Documented in Winter term 1992/1993\footnote{Urs von Matt, ‘Kassandra’ (1994) 22(1) \textit{SIGCUE bulletin} 26>}
			\pause
		\item \textbf{Testing Methodology:} External Program Side-effect Comparison
			\pause
		\item \textbf{Language:} Bash Shell \& MatLab \& Maple
			\pause
		\item \textbf{Maintained/Support:} Unknown (Assumed to be abandoned due to age)
	\end{enumerate}
\end{frame}
\begin{frame}
	\frametitle{TRY}
	\begin{enumerate}
		\setlength\itemsep{1em}
		\item \textbf{Author(s) \& Introduction:} Kenneth A Reek (Rochester Institute of Technlogy) - 1989 at Earliest\footnote{Kenneth A Reek, ‘The TRY System -or- How to Avoid Testing Student Programs’ (1989) 21(1) \textit{ACM SIGCSE Bulletin} 112}
			\pause
		\item \textbf{Testing Methodology:} External Program Side-effect Comparison
			\pause
		\item \textbf{Language:} C
			\pause
		\item \textbf{Maintained/Support:} Unknown (Assumed to be abandoned due to age)
	\end{enumerate}
\end{frame}

\section{Design}
\subsection{Design Requirements}
\begin{frame}
	\frametitle{Design Requirements}
	The design requirements for the solution will emphasise the following properties in accordance with the thesis goals:\\
	\begin{enumerate}
		\setlength\itemsep{0.5em}
		\item \textit{Accessible} - Users of the new software package should have the easiest possible experience in integrating automatic testing and grading into their courses
			\pause
		\item \textit{Familiar} - Users who have previously utilised Andrew Taylor autotest should not feel the new software package to be completely independent of the former 
			\pause
		\item \textit{Performant/Efficient} - The new software package should have the same, if not better performance than the original Andrew Taylor autotest. Benchmarking will be performed to verify this property
			\pause
		\item \textit{Maintainable} - The new software package should be adequately documented and support ease of maintainability and possible extension of features via the appropriate architectural decisions
	\end{enumerate}
\end{frame}
\begin{frame}
	\frametitle{Exclusions}
	\begin{enumerate}
		\setlength\itemsep{1em}
		\item \textit{Security} - We assume security to be out of scope as this is not a consideration in the existing Andrew Taylor autotest software package
			\pause
		\item Security considerations will greatly increase the complexity of the software and delivery of the software package before the conclusion of Thesis C is likely to be infeasible
			\pause
		\item \textit{Novelty} - We assume novelty to be out of scope as introducing a novel user experience is in conflict with the aforementioned design requirements
			\pause
		\item Users of the existing Andrew Taylor autotest may elect to not utilise the new software package if the transition is less than convenient
	\end{enumerate}
\end{frame}

\subsection{Proposed Design}
\begin{frame}[fragile]
	\frametitle{Approach 1}
	\begin{columns}
		\begin{column}{0.6\textwidth}
			\begin{enumerate}
				\setlength\itemsep{0.70em}
				\item Architectural approach similar to Harvard University check50
				\item check50 can be utilised as a ``baseline" for both performance and correctness testing of the solution
				\item Future Work and other relevant sections from the check50 paper can be utilised to inform implementation decisions for the solution
				\item \textbf{Main methodology to determine correctness of tested program:} External Program Side-effect Comparison
			\end{enumerate}
		\end{column}
		\begin{column}{0.35\textwidth}
			\centering
			\begin{tikzpicture}
				\node (Tester)[process]{Tester};
				\node (Test Runner)[process, below = 2em of Tester]{Test Runner};
				\node (Tests)[io, above = 2em of Tester, align=center]{Test Cases};
				\node (Program)[process, below = 2em of Test Runner]{Program};
				\draw [-stealth] (Tests) -- (Tester);
				\draw [stealth-stealth] (Tester) -- (Test Runner);
				\draw [stealth-stealth] (Test Runner) -- (Program);
			\end{tikzpicture}
		\end{column}
	\end{columns}
\end{frame}
\begin{frame}[fragile]
	\frametitle{Approach 2}
	\begin{columns}
		\begin{column}{0.6\textwidth}
			\begin{enumerate}
				\setlength\itemsep{0.70em}
				\item Architectural approach similar to existing Andrew Taylor autotest which has been widely accepted for introductory programming courses at UNSW
				\item autotest can be utilised as a ``baseline" for both performance and correctness testing of the solution
				\item \textbf{Main methodology to determine correctness of tested program:} External Program Side-effect Comparison
			\end{enumerate}
		\end{column}
		\begin{column}{0.35\textwidth}
			\centering
			\begin{tikzpicture}
				\node (Tester)[process]{Tester};
				\node (Program Output)[io, below = 2em of Tester, align=center]{Program\\Output};
				\node (Program)[process, below = 2em of Program Output]{Program};
				\node (Expected Output)[io, above = 2em of Tester, align=center]{Expected\\Output};
				\draw [-stealth] (Expected Output) -- (Tester);
				\draw [-stealth] (Program Output) -- (Tester);
				\draw [-stealth] (Program) -- (Program Output);
				\draw [-stealth, rounded corners] (Tester.east) -- (1.5,0) |- (Program.east);
			\end{tikzpicture}
		\end{column}
	\end{columns}
\end{frame}
\begin{frame}
	\frametitle{Chosen Approach}
	\begin{enumerate}
		\setlength\itemsep{1em}
		\item We select \textbf{Approach 2} over the alternative as it more compatible with the outlined design requirements 
			\pause
		\item We note that solutions implementing Approach 1 are more feature rich than Approach 2 but it must be considered that uptake of the solution is of higher importance than the features it provides
			\pause
		\item Features present from alternative approaches can also be added or removed in the interest of time but Approach 2 will provide a good starting point
			\pause
		\item Regardless of approach, we will implement all aforementioned technical debt management procedures
	\end{enumerate}
\end{frame}

\section{Schedule \& Conclusion}
\subsection{Schedule}
\begin{frame}
	\frametitle{Schedule}
	\begin{enumerate}
		\item Rest of thesis A:
		\begin{itemize}
			\item Continue inspection of Andrew Taylor autotest implementation
				\pause
			\item Complete Draft Design Document
				\pause
			\item Collect feedback on Design Document and make adjustments as necessary
				\pause 
		\end{itemize}
		\item Thesis B:
		\begin{itemize}
			\item Implement Core Main Module
				\pause
			\item Implement Core Testcase Parser Module
				\pause
			\item Implement Core Testcase Runner Module
				\pause
			\item Run correctness and performance testing on Parser and Runner
				\pause
		\end{itemize}
		\item Thesis C:
		\begin{itemize}
			\item Implement Core Testcase Program Correctness Module
				\pause
			\item Implement any extensions that have been deemed necessary by the Design Document
				\pause
			\item Run correctness and performance testing on complete package and make final adjustments
		\end{itemize}
	\end{enumerate}
\end{frame}
\subsection{Summary}
\begin{frame}
	\frametitle{Summary}
	We have covered:
	\begin{itemize}
		\item Student enrolments in Introductory Programming courses at UNSW, Andrew Taylor autotest, Technical Debt
			\pause
		\item The thesis statement
			\pause
		\item Code Marking in Introductory Courses
			\pause
		\item Approaches to Automated testing and marking
			\pause
		\item Existing and historic solutions for Automated testing and marking, deeper look into the most relevant solutions
			\pause
		\item Design requirements and exclusions
			\pause
		\item Approaches to the architecture of the solution
			\pause
		\item Development schedule of the solution over all thesis periods
			\pause
	\end{itemize}
	\textit{\textbf{Thank you for attending!} Questions?}
\end{frame}
\end{document}
